\documentclass{beamer}
\usepackage{graphics}
\usepackage{url}
\usepackage{ulem}
\usepackage{beamerthemesplit}
\usepackage{hyperref}
\usepackage{wrapfig}
\usepackage[spanish,activeacute]{babel}
\usepackage[utf8]{inputenc}
\usepackage{listings}
\usepackage{color}
\usetheme{Warsaw}
\usepackage{pstricks}

%\pgfdeclareimage[height=1.3cm]{logo-izq}{img/alma.png}
%\pgfdeclareimage[height=1.3cm]{logo-der}{img/alma-off.png}
%\logo{\pgfuseimage{logo-der}}
%\setbeamertemplate{sidebar left}
%{
%\logo{\pgfuseimage{logo-izq}}
%\vfill %pone la imágen en la esquina inferior izquierda
%\rlap{\hskip0.1cm\insertlogo} %inserta la imágen
%\vskip15pt
%}


\title{Rotative Workforce Scheduling Problem}
\subtitle{Presentación Final}
\author{Jonathan Antognini C.\\
		Luis Casanova S.
}
\institute[]{Universidad Técnica Federico Santa María}
\date{\today}

\begin{document}
    \frame{\titlepage}
    \frame{\tableofcontents}
	\section{Introducción}
		\frame
{
\frametitle{Introducción}
Rotating Workforce Scheduling, es un problema que busca realizar una planificación de turnos de trabajo para un conjunto 
de empleados. Esta planificación debe ser cíclica, es decir, que al final de cada período de planificación se realiza
una rotación entre los turnos asignados a cada empleado, esto sujeto a restricciones específicas.

\begin{center}
	\begin{tabular}{|l|l|l|l|}
	        \hline
	        Lu & Ma & Mi & Ju \\
	        \hline
	        A & A & D & - \\
	        \hline
	        D & D & N & N \\
	        \hline
	        - & - & A & N \\
	        \hline
	\end{tabular}
\end{center}

}

	
	\section{Implementaciones}
		\subsection{Estructura de datos}
		\frame{
\frametitle{Estructura de datos}
Los datos se obtuvieron de los input (Example*.txt), archivos que contenían los datos necesarios para poder definir el 
problema. Los datos son:

\begin{itemize}
        \item $w$: largo de la planificación. Todos los input tenían un $w=7$, lo que corresponde a planificar de lunes 
		a domingo.
        \item $n$: número de empleados.
        \item $m$: número de turnos más día libre. En la mayoría de los input $m$ valía 4 (3 turnos + 1 día libre).
        \item $A$: vector de largo m donde se indican los diferentes tipos de turnos. En la mayoría de los casos v 
		contenía $D,A,N,-$.
\end{itemize}
}

\frame{
\begin{itemize}
        \item $R$: matriz de requerimientos de turnos por día, ya que hay $w$ días, y $m-1$ día son turnos, la matriz 
		tiene dimensión $R_{(m-1)xw}$
        \item $MAXS$: Vector de largo m donde por cada turno se indica el máximo de turnos o días libres consecutivos 
		permitidos.
        \item $MINS$: Vector de largo m donde por cada turno se indica el mínimo de turnos o días libres consecutivos 
		permitidos.
        \item $MAXW$: número máximo de días consecutivos trabajados.
        \item $MINW$: número mínimo de días consecutivos trabajados.
        \item $C2$: matriz con secuencias de turnos no permitidas de largo 2.
	\item $C3$: matriz con secuencias de turnos no permitidas de largo 3.
\end{itemize}
}

		\subsection{Foward checking + Graph BackJumping}
		\frame
{
\frametitle{MFC + GBJ: Representación}
Para implementar el algoritmo se crea el arreglo $solucion$ largo $w \times n$.\\
\begin{table}[H]
\begin{center}
\begin{tabular}{|c|c|c|c|c|c|c|c|}
\hline  0& 1 & 2&3&4&5&6&...  \\ 
\hline 
\end{tabular} 
\end{center}
\end{table}

Para una mejor comprensión se puede ver el arreglo como matriz, en donde las filas son los empleados y los dias las columnas, los valores al interior 
son a los índices correspondientes al arreglo $solucion$, y en el orden en que son instanciados.
\begin{table}[H]
\begin{center}
\begin{tabular}{|c|c|c|}
\hline  0& 6 & 12  \\ 
\hline  1& 7& 13 \\ 
\hline  2& 8 & 14  \\ 
\hline  3&9 & 15\\ 
\hline  4& 10 & 16  \\ 
\hline  5& 11& 17  \\ 
\hline 
\end{tabular} 
\end{center}
\end{table}
}
\frame
{
\frametitle{MFC y GBJ}
Funciones $MFC$:\\
Se verifican las restricciones,basta encontrar basta encontrar un camino factible, se eliminan los dominios futuros incompatibles.\\ \vspace{14 pt}
Funciones restricción para el $GBJ$:\\
Se verifican las restricciones, en caso de no cumplirse se retorna 1 para realizar el salto al nodo más recientemente instanciado y conectado, este se encuentra en la lista padres de cada nodo.\\

}

\frame{
\frametitle{MFC+GBJ}

Item a grandes rasgos los pasos realizados son:
\begin{enumerate}
\item Se guardan los padres de cada nodo, en una lista.
\item Realiza la asignación del dominio.
\item Realiza función $MFC$ para filtrar dominios.
\item Realiza función $consistente$ que realiza los chequeos de restricciones para el GBJ.
\item Analiza si debe realizar un salto, si es así es al mayor valor correspondiente a la lista padre del nodo actual.
\item De ser así realiza el salto y se reestablecen los dominios y la solución.
\item Si no se vuelve al punto 2, hasta llegar a la solución final.

\end{enumerate}

}

		\subsection{Greedy + Hill Climbing}
		\frame{
\frametitle{Greedy + HC: Representación}
Considerando $w=4$ y $n=3$:
\begin{center}
	\begin{tabular}{|l|l|l|l|}
	        \hline
	        Lu & Ma & Mi & Ju \\
	        \hline
	        A & A & D & - \\
	        \hline
	        D & D & N & N \\
	        \hline
	        - & - & A & N \\
	        \hline
	\end{tabular}
\end{center}

Para comprobación de restricciones, se representó de la siguiente forma:
\begin{center}
	\begin{tabular}{|l|l|l|l|l|l|l|l|l|l|l|l|}
	        \hline
	        Lu & Ma & Mi & Ju & Lu & Ma & Mi & Ju & Lu & Ma & Mi & Ju \\
	        \hline
	        A & A & D & - & D & D & N & N & - & - & A & N \\
	        \hline
	\end{tabular}
\end{center}
}

\frame{
\frametitle{Greedy}
Para poder definir un algoritmo Greedy correctamente es necesario especificar:
\begin{itemize}
        \item Solución inicial: solución obtenida mediante greedy.
        \item Función objetivo: minimizar la cantidad de penalizaciones hechas debido a restricciones blandas insatisfechas.
        \item Punto de partida: el día en donde se quiera empezar a planificar, es decir, se le entregará un día entre 1 y W.
        \item Función miope: para el día i, se le asigna al primer trabajador disponible el turno requerido, de tal 
		forma que la diferencia entre la cantidad de empleados necesarios en dicho turno se minimize.
\end{itemize}
}

\frame{
\frametitle{Hill Climbing}
\begin{itemize}
        \item Número de restart: definido como constante (se cambiaba por cada instancia).
        \item Solución inicial: solución obtenida mediante greedy.
        \item Función objetivo: minimizar la cantidad de penalizaciones hechas debido a restricciones blandas insatisfechas.
\end{itemize}
}

\frame{
\begin{itemize}
        \item Movimiento: swaps de turnos entre turnos de un día. Por ejemplo:
                \begin{center}
                	\begin{tabular}{|l|l|l|l|}
                 	       \hline
                 	       Lu & Ma & Mi & Ju \\
                 	       \hline
                 	       A & A & D & - \\
                 	       \hline
                  	      D & D & N & N \\
                  	      \hline
                   	     - & - & A & N \\
                        	\hline
                	\end{tabular}
                \end{center}

                Al aplicar el movimiento y generar el primer vecino, se hace un swap de la casilla 1,1, 
		con la casilla 2,1, quedando:
                \begin{center}
			\begin{tabular}{|l|l|l|l|}
           	             \hline
           	             Lu & Ma & Mi & Ju \\
           	             \hline
           	             D & A & D & - \\
           	             \hline
           	             A & D & N & N \\
           	             \hline
           	             - & - & A & N \\
           	             \hline
			\end{tabular}
                \end{center}
\end{itemize}
}

\frame{
\frametitle{Greedy + HC}
\begin{itemize}
        \item Se inicializa una solución vacía.
        \item Se pasa la solución vacía al greedy, y además un día de comienzo. El resultado de esta operación genera 
		una solución que respeta las restricciones duras ($R$).
        \item La solución del greedy es la entrada ahora para el hill climbing.
        \item Se realiza el movimiento sobre la solución actual.
        \item Si el algoritmo no encuentra un mejor vecino, entonces se hace un restart.
        \item Cuando se termina la cantidad de restart, el algoritmo acaba y entrega la mejor solución encontrada.
\end{itemize}
}


	\section{Resultads}
		\frame
{
\frametitle{Resultados}
Los resultados encontrados mediante greedy+hc fueron:
\begin{center}
	\begin{tabular}{|l|l|}
	        \hline
	        Instancia       & Valor función objetivo        \\
	        \hline
	        Example1.txt    & 35                            \\
	        Example2.txt    & 52                            \\
	        Example3.txt    & 79                            \\
	        Example4.txt    & 32                            \\
	        Example5.txt    & 46                            \\
	        Example6.txt    & 22                            \\
	        Example7.txt    & 289                           \\
	        Example8.txt    & 71                            \\
	        \hline
	\end{tabular}
\end{center}
}
\frame
{
\frametitle{Resultados}
Mediante MFC+GBJ no se encontraron resultados.\\\vspace{14 pt}
Esto debido a la gran complejidad del algoritmo y que además cuenta con 2 técnicas,  esto complico la realización del debugueo.\\\vspace{14 pt}
Al tomar todas las restricciones como duras se reduce de manera considerable la cantidad de soluciones finales aceptadas, por lo
que complica aún más el encontrar soluciones.\\\vspace{14 pt}
Las instancias de los problemas son bastante grandes, se realizan instancias más pequeñas, pero no se puede comprobar fácilmente que estas tengan solución.\\

}


	\section{Conclusiones}
		\frame
{
\frametitle{Conclusiones}
\begin{itemize}
	\item Aprendizaje
	\item Técnica completa
	\item Técnica incompleta
	\item Generales
\end{itemize}
}
	

\end{document}
