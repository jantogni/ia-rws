\frame{
\frametitle{Estructura de datos}
Los datos se obtuvieron de los input (Example*.txt), archivos que contenían los datos necesarios para poder definir el 
problema. Los datos son:

\begin{itemize}
        \item $w$: largo de la planificación. Todos los input tenían un $w=7$, lo que corresponde a planificar de lunes 
		a domingo.
        \item $n$: número de empleados.
        \item $m$: número de turnos más día libre. En la mayoría de los input $m$ valía 4 (3 turnos + 1 día libre).
        \item $A$: vector de largo m donde se indican los diferentes tipos de turnos. En la mayoría de los casos v 
		contenía $D,A,N,-$.
\end{itemize}
}

\frame{
\begin{itemize}
        \item $R$: matriz de requerimientos de turnos por día, ya que hay $w$ días, y $m-1$ día son turnos, la matriz 
		tiene dimensión $R_{(m-1)xw}$
        \item $MAXS$: Vector de largo m donde por cada turno se indica el máximo de turnos o días libres consecutivos 
		permitidos.
        \item $MINS$: Vector de largo m donde por cada turno se indica el mínimo de turnos o días libres consecutivos 
		permitidos.
        \item $MAXW$: número máximo de días consecutivos trabajados.
        \item $MINW$: número mínimo de días consecutivos trabajados.
        \item $C2$: matriz con secuencias de turnos no permitidas de largo 2.
	\item $C3$: matriz con secuencias de turnos no permitidas de largo 3.
\end{itemize}
}
