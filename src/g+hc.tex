\frame{
\frametitle{Greedy + HC: Representación}
Considerando $w=4$ y $n=3$:
\begin{center}
	\begin{tabular}{|l|l|l|l|}
	        \hline
	        Lu & Ma & Mi & Ju \\
	        \hline
	        A & A & D & - \\
	        \hline
	        D & D & N & N \\
	        \hline
	        - & - & A & N \\
	        \hline
	\end{tabular}
\end{center}

Para comprobación de restricciones, se representó de la siguiente forma:
\begin{center}
	\begin{tabular}{|l|l|l|l|l|l|l|l|l|l|l|l|}
	        \hline
	        Lu & Ma & Mi & Ju & Lu & Ma & Mi & Ju & Lu & Ma & Mi & Ju \\
	        \hline
	        A & A & D & - & D & D & N & N & - & - & A & N \\
	        \hline
	\end{tabular}
\end{center}
}

\frame{
\frametitle{Greedy}
Para poder definir un algoritmo Greedy correctamente es necesario especificar:
\begin{itemize}
        \item Solución inicial: solución obtenida mediante greedy.
        \item Función objetivo: minimizar la cantidad de penalizaciones hechas debido a restricciones blandas insatisfechas.
        \item Punto de partida: el día en donde se quiera empezar a planificar, es decir, se le entregará un día entre 1 y W.
        \item Función miope: para el día i, se le asigna al primer trabajador disponible el turno requerido, de tal 
		forma que la diferencia entre la cantidad de empleados necesarios en dicho turno se minimize.
\end{itemize}
}

\frame{
\frametitle{Hill Climbing}
\begin{itemize}
        \item Número de restart: definido como constante (se cambiaba por cada instancia).
        \item Solución inicial: solución obtenida mediante greedy.
        \item Función objetivo: minimizar la cantidad de penalizaciones hechas debido a restricciones blandas insatisfechas.
\end{itemize}
}

\frame{
\begin{itemize}
        \item Movimiento: swaps de turnos entre turnos de un día. Por ejemplo:
                \begin{center}
                	\begin{tabular}{|l|l|l|l|}
                 	       \hline
                 	       Lu & Ma & Mi & Ju \\
                 	       \hline
                 	       A & A & D & - \\
                 	       \hline
                  	      D & D & N & N \\
                  	      \hline
                   	     - & - & A & N \\
                        	\hline
                	\end{tabular}
                \end{center}

                Al aplicar el movimiento y generar el primer vecino, se hace un swap de la casilla 1,1, 
		con la casilla 2,1, quedando:
                \begin{center}
			\begin{tabular}{|l|l|l|l|}
           	             \hline
           	             Lu & Ma & Mi & Ju \\
           	             \hline
           	             D & A & D & - \\
           	             \hline
           	             A & D & N & N \\
           	             \hline
           	             - & - & A & N \\
           	             \hline
			\end{tabular}
                \end{center}
\end{itemize}
}

\frame{
\frametitle{Greedy + HC}
\begin{itemize}
        \item Se inicializa una solución vacía.
        \item Se pasa la solución vacía al greedy, y además un día de comienzo. El resultado de esta operación genera 
		una solución que respeta las restricciones duras ($R$).
        \item La solución del greedy es la entrada ahora para el hill climbing.
        \item Se realiza el movimiento sobre la solución actual.
        \item Si el algoritmo no encuentra un mejor vecino, entonces se hace un restart.
        \item Cuando se termina la cantidad de restart, el algoritmo acaba y entrega la mejor solución encontrada.
\end{itemize}
}
