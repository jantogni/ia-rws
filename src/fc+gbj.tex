\frame
{
\frametitle{Representación}
Para implementar el algoritmo se crea el arreglo $solucion$ largo $w \times n$.\\
\begin{table}[H]
\begin{center}
\begin{tabular}{|c|c|c|c|c|c|c|c|}
\hline  0& 1 & 2&3&4&5&6&...  \\ 
\hline 
\end{tabular} 
\end{center}
\end{table}

Para una mejor comprensión se puede ver el arreglo como matriz, en donde las filas son los empleados y los dias las columnas, los valores al interior son a los índices correspondientes al arreglo $solucion$, y en el orden en que son instanciados.
\begin{table}[H]
\begin{center}
\begin{tabular}{|c|c|c|}
\hline  0& 6 & 12  \\ 
\hline  1& 7& 13 \\ 
\hline  2& 8 & 14  \\ 
\hline  3&9 & 15\\ 
\hline  4& 10 & 16  \\ 
\hline  5& 11& 17  \\ 
\hline 
\end{tabular} 
\end{center}
\end{table}
}
\frame
{
\frametitle{Foward Checking}
Funciones mfc:\\
Se verifican las restricciones y se eliminan los dominios futuros incompatibles.
Funciones restricción gbj;\\
Se verifican las restricciones, en caso de no cumplirse se devuelve la cantidad de nodos que se debe devolver.\\
Gbj:\\
Item a grandes rasgos los pasos realizados son:
\begin{itemize}
\item Realiza la asignación del dominio
\item 
\end{itemize}

}
