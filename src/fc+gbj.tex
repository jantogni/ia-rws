\frame
{
\frametitle{Representación}
Para implementar el algoritmo se crea el arreglo $solucion$ largo $w \times n$.\\
\begin{table}[H]
\begin{center}
\begin{tabular}{|c|c|c|c|c|c|c|c|}
\hline  0& 1 & 2&3&4&5&6&...  \\ 
\hline 
\end{tabular} 
\end{center}
\end{table}

Para una mejor comprensión se puede ver el arreglo como matriz, en donde las filas son los empleados y los dias las columnas, los valores al interior son a los índices correspondientes al arreglo $solucion$, y en el orden en que son instanciados.
\begin{table}[H]
\begin{center}
\begin{tabular}{|c|c|c|}
\hline  0& 6 & 12  \\ 
\hline  1& 7& 13 \\ 
\hline  2& 8 & 14  \\ 
\hline  3&9 & 15\\ 
\hline  4& 10 & 16  \\ 
\hline  5& 11& 17  \\ 
\hline 
\end{tabular} 
\end{center}
\end{table}
}
\frame
{
\frametitle{MFC y GBJ}
Funciones $MFC$:\\
Se verifican las restricciones,basta encontrar basta encontrar un camino factible, se eliminan los dominios futuros incompatibles.\\ \vspace{14 pt}
Funciones restricción para el $GBJ$:\\
Se verifican las restricciones, en caso de no cumplirse se retorna 1 para realizar el salto al nodo más recientemente instanciado y conectado, este se encuentra en la lista padres de cada nodo.\\

}

\frame{
\frametitle{MFC+GBJ}

Item a grandes rasgos los pasos realizados son:
\begin{enumerate}
\item Se guardan los padres de cada nodo, en una lista.
\item Realiza la asignación del dominio.
\item Realiza función $MFC$ para filtrar dominios.
\item Realiza función $consistente$ que realiza los chequeos de restricciones para el GBJ.
\item Analiza si debe realizar un salto, si es así es al mayor valor correspondiente a la lista padre del nodo actual.
\item De ser así realiza el salto y se reestablecen los dominios y la solución.
\item Si no se vuelve al punto 2, hasta llegar a la solución final.

\end{enumerate}

}